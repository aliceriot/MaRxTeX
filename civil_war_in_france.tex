\documentclass{book}

\begin{document}

\title{Civil War in France}
\author{Karl Marx}

\maketitle

\chapter{Introduction}

Thanks to the economic and political development of France since [the
French Revolution of] 1789, for 50 years the position of Paris has been
such that no revolutions could break out there without assuming
a proletarian character, that is to say, the proletariat, which had bought
victory with its blood, would advance its own demands after victory. These
demands were more or less unclear and even confused, corresponding to the
state of evolution reached by the workers of Paris at the particular
period, but in the last resort they all amounted to the abolition of the
class antagonism between capitalist and workers. It is true that no one
knew how this was to be brought about. But the demand itself, however
indefinite it still was in its formulation, contained a threat to the
existing order of society; the workers who put it forward were still
armed; therefore the disarming of the workers was the first commandment
for the bourgeois at the helm of the state. Hence, after every revolution
won by the workers, a new struggle, ending with the defeat of the workers.

This happened for the first time in 1848. The liberal bourgeoisie of the
parliamentary opposition held banquets for securing reform of the
franchise, which was to ensure supremacy for their party. Forced more and
more, in their struggle with the government, to appeal to the people, they
had to allow the radical and republican strata of the bourgeoisie and
petty bourgeoisie gradually to take the lead. But behind these stood the
revolutionary workers, and since 1830,[A] these had acquired far more
political independence than the bourgeoisie, and even the republicans,
suspected. At the moment of the crisis between the government and the
opposition, the workers opened battle on the streets; [King] Louis
Philippe vanished, and with him the franchise reform; and in its place
arose the republic, and indeed one which the victorious workers themselves
designated as a “social” republic. No one, however, was clear as to what
this social republic was to imply; not even the workers themselves. But
they now had arms in their hands, and were a power in the state.
Therefore, as soon as the bourgeois republicans in control felt something
like firm ground under their feet, their first aim was to disarm the
workers. This took place by driving them into the insurrection of June
1848 by direct breach of faith, by open defiance and the attempt to banish
the unemployed to a distant province. The government had taken care to
have an overwhelming superiority of force. After five days’ heroic
struggle, the workers were defeated. And then followed a blood-bath of the
defenceless prisoners, the likes of which as not been seen since the days
of the civil wars which ushered in the downfall of the Roman republic. It
was the first time that the bourgeoisie showed to what insane cruelties of
revenge it will be goaded the moment the proletariat dares to take its
stand against them as a separate class, with its own interests and
demands. And yet 1848 was only child’s play compared with their frenzy in
1871.

Punishment followed hard at heel. If the proletariat was not yet able to
rule France, the bourgeoisie could no longer do so. At least not at that
period, when the greater part of it was still monarchically inclined, and
it was divided into three dynastic parties [Legitimists, Orleanists and
Bonapartists] and a fourth republican party. Its internal dissensions
allowed the adventurer Louis Bonaparte to take possession of all the
commanding points – army, police, administrative machinery – and, on
December 2, 1851,[B] to explode the last stronghold of the bourgeoisie,
the National Assembly. The Second Empire opened the exploitation of France
by a gang of political and financial adventurers, but at the same time
also an industrial development such as had never been possible under the
narrow-minded and timorous system of Louis Philippe, with its exclusive
domination by only a small chapter of the big bourgeoisie. Louis Bonaparte
took the political power from the capitalists under the pretext of
protecting them, the bourgeoisie, from the workers, and on the other hand
the workers from them; but in return his rule encouraged speculation and
industrial activity – in a word the rise and enrichment of the whole
bourgeoisie to an extent hitherto unknown. To an even greater extent, it
is true, corruption and mass robbery developed, clustering around the
imperial court, and drawing their heavy percentages from this enrichment.

But the Second Empire was the appeal to the French chauvinism, the demand
for the restoration of the frontiers of the First Empire, which had been
lost in 1814, or at least those of the First Republic.[C] A French empire
within the frontiers of the old monarchy and, in fact, within the even
more amputated frontiers of 1815 – such a thing was impossible for any
long duration of time. Hence the necessity for brief wars and extension of
frontiers. But no extension of frontiers was so dazzling to the
imagination of the French chauvinists as the extension to the German left
bank of the Rhine. One square mile on the Rhine was more to them than ten
in the Alps or anywhere else. Given the Second Empire, the demand for the
restoration to France of the left bank of the Rhine, either all at once or
piecemeal, was merely a question of time. The time came with the
Austro-Prussian War of 1866; cheated of the anticipated “territorial
compensation” by Bismarck, and by his own over-cunning, hesitating policy,
there was now nothing left for Napoleon but war, which broke out in 1870
and drove him first to Sedan, and then to Wilhelmshohe [prison].

The inevitable result was the Paris Revolution of September 4, 1870. The
empire collapsed like a house of cards, and the republic was again
proclaimed. But the enemy was standing at the gates [of Paris]; the armies
of the empire were either hopelessly beleaguered in Metz or held captive
in Germany. In this emergency the people allowed the Paris Deputies to the
former legislative body to constitute themselves into a “Government of
National Defence.” This was the more readily conceded, since, for the
purpose of defence, all Parisians capable of bearing arms had enrolled in
the National Guard and were armed, so that now the workers constituted
a great majority. But almost at once the antagonism between the almost
completely bourgeois government and the armed proletariat broke into open
conflict. On October 31, workers’ battalions stormed the town hall, and
captured some members of the government. Treachery, the government’s
direct breach of its undertakings, and the interventions of some
petty-bourgeois battalions set them free again, and in order not to
occasion the outbreak of civil war inside a city which was already
beleaguered by a foreign power, the former government was left in office.

At last on January 28, 1871, Paris, almost starving, capitulated but with
honors unprecedented in the history of war. The forts were surrendered,
the outer wall disarmed, the weapons of the regiments of the line and of
the Mobile Guard were handed over, and they themselves considered
prisoners of war. But the National Guard kept its weapons and guns, and
only entered into an armistice with the victors, who themselves did not
dare enter Paris in triumph. They only dared to occupy a tiny corner of
Paris, which, into the bargain, consisted partly of public parks, and even
this they only occupied for a few days! And during this time they, who had
maintained their encirclement of Paris for 131 days, were themselves
encircled by the armed workers of Paris, who kept a sharp watch that no
“Prussian” should overstep the narrow bounds of the corner ceded to the
foreign conquerors. Such was the respect which the Paris workers inspired
in the army before which all the armies of the empire had laid down their
arms; and the Prussian Junkers, who had come to take revenge at the very
centre of the revolution, were compelled to stand by respectfully, and
salute just precisely this armed revolution!

During the war the Paris workers had confined themselves to demanding the
vigorous prosecution of the fight. But now, when peace had come after the
capitulation of Paris,[D] now, Thiers, the new head of government, was
compelled to realize that the supremacy of the propertied classes – large
landowners and capitalists – was in constant danger so long as the workers
of Paris had arms in their hands. His first action was to attempt to
disarm them. On March 18, he sent troops of the line with orders to rob
the National Guard of the artillery belonging to it, which had been
constructed during the siege of Paris and had been paid for by public
subscription. The attempt failed; Paris mobilized as one man in defence of
the guns, and war between Paris and the French government sitting at
Versailles was declared. On March 26 the Paris Commune was elected and on
March 28 it was proclaimed. The Central Committee of the National Guard,
which up to then had carried on the government, handed in its resignation
to the National Guard, after it had first decreed the abolition of the
scandalous Paris “Morality Police.” On March 30 the Commune abolished
conscription and the standing army, and declared that the National Guard,
in which all citizens capable of bearing arms were to be enrolled, was to
be the sole armed force. It remitted all payments of rent for dwelling
houses from October 1870 until April, the amounts already paid to be
reckoned to a future rental period, and stopped all sales of articles
pledged in the municipal pawnshops. On the same day the foreigners elected
to the Commune were confirmed in office, because “the flag of the Commune
is the flag of the World Republic.”

On April 1 it was decided that the highest salary received by any employee
of the Commune, and therefore also by its members themselves, might not
exceed 6,000 francs. On the following day the Commune decreed the
separation of the Church from the State, and the abolition of all state
payments for religious purposes as well as the transformation of all
Church property into national property; as a result of which, on April 8,
a decree excluding from the schools all religious symbols, pictures,
dogmas, prayers – in a word, “all that belongs to the sphere of the
individual’s conscience” – was ordered to be excluded from the schools,
and this decree was gradually applied. On the 5th, in reply to the
shooting, day after day, of the Commune’s fighters captured by the
Versailles troops, a decree was issued for imprisonment of hostages, but
it was never carried into effect. On the 6th, the guillotine was brought
out by the 137th battalion of the National Guard, and publicly burnt, amid
great popular rejoicing. On the 12th, the Commune decided that the Victory
Column on the Place Vendôme, which had been cast from guns captured by
Napoleon after the war of 1809, should be demolished as a symbol of
chauvinism and incitement to national hatred. This decree was carried out
on May 16. On April 16 the Commune ordered a statistical tabulation of
factories which had been closed down by the manufacturers, and the working
out of plans for the carrying on of these factories by workers formerly
employed in them, who were to be organized in co-operative societies, and
also plans for the organization of these co-operatives in one great union.
On the 20th the Commune abolished night work for bakers, and also the
workers’ registration cards, which since the Second Empire had been run as
a monopoly by police nominees – exploiters of the first rank; the issuing
of these registration cards was transferred to the mayors of the 20
arrondissements of Paris. On April 30, the Commune ordered the closing of
the pawnshops, on the ground that they were a private exploitation of
labor, and were in contradiction with the right of the workers to their
instruments of labor and to credit. On May 5 it ordered the demolition of
the Chapel of Atonement, which had been built in expiation of the
execution of Louis XVI.

Thus, from March 18 onwards the class character of the Paris movement,
which had previously been pushed into the background by the fight against
the foreign invaders, emerged sharply and clearly. As almost without
exception, workers, or recognized representatives of the workers, sat in
the Commune, its decision bore a decidedly proletarian character. Either
they decreed reforms which the republican bourgeoisie had failed to pass
solely out of cowardice, but which provided a necessary basis for the free
activity of the working class – such as the realization of the principle
that in relation to the state, religion is a purely private matter – or
they promulgated decrees which were in the direct interests of the working
class and to some extent cut deeply into the old order of society. In
a beleaguered city, however, it was possible at most to make a start in
the realization of all these measures. And from the beginning of May
onwards all their energies were taken up by the fight against the
ever-growing armies assembled by the Versailles government.

On April 7, the Versailles troops had captured the Seine crossing at
Neuilly, on the western front of Paris; on the other hand, in an attack on
the southern front on the 11th they were repulsed with heavy losses by
General Eudes. Paris was continually bombarded and, moreover, by the very
people who had stigmatized as a sacrilege the bombardment of the same city
by the Prussians. These same people now begged the Prussian government for
the hasty return of the French soldiers taken prisoner at Sedan and Metz,
in order that they might recapture Paris for them. From the beginning of
May the gradual arrival of these troops gave the Versailles forces
a decided ascendancy. This already became evident when, on April 23,
Thiers broke off the negotiations for the exchange, proposed by Commune,
of the Archbishop of Paris [Georges Darboy] and a whole number of other
priests held hostages in Paris, for only one man, Blanqui, who had twice
been elected to the Commune but was a prisoner in Clairvaux. And even more
from the changed language of Thiers; previously procrastinating and
equivocal, he now suddenly became insolent, threatening, brutal. The
Versailles forces took the redoubt of Moulin Saquet on the southern front,
on May 3; on the 9th, Fort Issy, which had been completely reduced to
ruins by gunfire; and on the 14th, Fort Vanves. On the western front they
advanced gradually, capturing the numerous villages and buildings which
extended up to the city wall, until they reached the main wall itself; on
the 21st, thanks to treachery and the carelessness of the National Guards
stationed there, they succeeded in forcing their way into the city. The
Prussians who held the northern and eastern forts allowed the Versailles
troops to advance across the land north of the city, which was forbidden
ground to them under the armistice, and thus to march forward and attack
on a long front, which the Parisians naturally thought covered by the
armistice, and therefore held only with weak forces. As a result of this,
only a weak resistance was put up in the western half of Paris, in the
luxury city proper; it grew stronger and more tenacious the nearer the
incoming troops approached the eastern half, the real working class city.

It was only after eight days’ fighting that the last defender of the
Commune were overwhelmed on the heights of Belleville and Menilmontant;
and then the massacre of defenceless men, women, and children, which had
been raging all through the week on an increasing scale, reached its
zenith. The breechloaders could no longer kill fast enough; the vanquished
workers were shot down in hundred by mitrailleuse fire [over 30,000
citizens of Paris were massacred]. The “Wall of the Federals” [aka Wall of
the Communards] at the Pere Lachaise cemetery, where the final mass murder
was consummated, is still standing today, a mute but eloquent testimony to
the savagery of which the ruling class is capable as soon as the working
class dares to come out for its rights. Then came the mass arrests [38,000
workers arrested]; when the slaughter of them all proved to be impossible,
the shooting of victims arbitrarily selected from the prisoners’ ranks,
and the removal of the rest to great camps where they awaited trial by
courts-martial. The Prussian troops surrounding the northern half of Paris
had orders not to allow any fugitives to pass; but the officers often shut
their eyes when the soldiers paid more obedience to the dictates of
humanity than to those of the General Staff; particularly, honor is due to
the Saxon army corps, which behaved very humanely and let through many
workers who were obviously fighters for the Commune.

\emph{Frederick Engels}

London, on the 20th anniversary of the Paris Commune, March 18, 1891.

\chapter{The Beginning of the Franco Prussian War}

In the Inaugural Address of the International Working Men’s Association,
of November 1864, we said:

\begin{quote} 

``If the emancipation of the working classes requires their fraternal
concurrence, how are they to fulfill that great mission with a foreign
policy in pursuit of criminal designs, playing upon national prejudices,
and squandering in piratical wars the people’s blood and treasure?''

\end{quote}

We defined the foreign policy aimed at by the International in these
words:

\begin{quote}

``Vindicate the simple laws of morals and justice, which ought to govern
the relations of private individuals, as the laws paramount of the
intercourse of nations.”

\end{quote}

No wonder that Louis Bonaparte, who usurped power by exploiting the war of
classes in France, and perpetuated it by periodical wars abroad, should,
from the first, have treated the International as a dangerous foe. On the
eve of the plebiscite \footnote{A plebiscite is a direct vote by an
electorate of a nation to decide a question of national importance, such
as governmental policy. Conducted by Napoleon III in May 1870 the
questions were so worded that it was impossible to express disapproval of
the policy of the Second Empire without declaring opposition to all
democratic reforms for the working class. The chapters of the First
International in France argued that their members should not participate
in the vote. On the eve of the plebiscite members of the Paris Federation
were arrested on a charge of conspiring against Napoleon III. This pretext
was further used by the government to launch a campaign of persecution of
the members of the International throughout France. At the trial of the
Paris Federation members (June 22 to July 5, 1870), the charge of
conspiracy was clearly exposed as without any basis. Nevertheless a number
of the International’s members were sentenced to imprisonment based solely
on their socialistic beliefs. The working class of France responded to
these political persecutions with mass protests.} he ordered a raid on the
members of the Administrative Committee of the International Working Men’s
Association throughout France, at Paris, Lyons, Rouen, Marseilles, Brest,
etc., on the pretext that the International was a secret society dabbling
in a complot for his assassination, a pretext soon after exposed in its
full absurdity by his own judges. What was the real crime of the French
branches of the International? They told the French people publicly and
emphatically that voting the plebiscite was voting despotism at home and
war abroad. It has been, in fact, their work that in all the great towns,
in all the industrial centres of France, the working class rose like one
man to reject the plebiscite. Unfortunately, the balance was turned by the
heavy ignorance of the rural districts. The stock exchanges, the cabinets,
the ruling classes, and the press of Europe celebrated the plebiscite as
a signal victory of the French emperor over the French working class; and
it was the signal for the assassination, not of an individual, but of
nations.

The war plot of July 19 1870 \footnote{The date when Napoleon III declared
war on Prussia.} is but an amended edition of the coup d’etat of December
1851. At first view, the thing seemed so absurd that France would not
believe in its real good earnest. It rather believed the deputy denouncing
the ministerial war talk as a mere stock-jobbing trick. When, on July 15,
war was at last officially announced to the Corps Legislatif, the whole
Opposition refused to vote the preliminary subsidies – even Thiers branded
it as “detestable”; all the independent journals of Paris condemned it,
and, wonderful to relate, the provincial press joined in almost
unanimously.

Meanwhile, the Paris members of the International had again set to work.
In the Reveil of July 12, they published their manifesto “to the Workmen
of all Nations,” from which we extract the following few passages:

\begin{quote}

“Once more,” they say, “on the pretext of European equilibrium, of
national honor, the peace of the world is menaced by political ambitions.
French, German, Spanish workmen! Let our voices unite in one cry of
reprobation against war!


``War for a question of preponderance or a dynasty can, in the eyes of
workmen, be nothing but a criminal absurdity. In answer to the warlike
proclamations of those who exempt themselves from the blood tax, and find
in public misfortunes a source of fresh speculations, we protest, we who
want peace, labor, and liberty!

``Brothers in Germany! Our division would only result in the complete
triumph of the despotism on both sides of the Rhine...

``Workmen of all countries! Whatever may for the present become of our
common efforts, we, the members of the International Working Men’s
Association, who know of no frontiers, we send you, as a pledge of
indissoluble solidarity, the good wishes and the salutations of the
workmen of France.”

\end{quote}

This manifesto of our Paris chapter was followed by numerous similar
French addresses, of which we can here only quote the declaration of
Neuilly-sur-Seine, published in the Marseillaise of July 22:

\begin{quote}

``The war, is it just? No! The war, is it national? No! It is merely
dynastic. In the name of humanity, or democracy, and the true interests of
France, we adhere completely and energetically to the protestation of the
International against the war.''

\end{quote}

These protestations expressed the true sentiments of the French working
people, as was soon shown by a curious incident. The Band of the 10th of
December, first organized under the presidency of Louis Bonaparte, having
been masqueraded into blouses (i.e., to appear as common workers) and let
loose on the streets of Paris, there to perform the contortions of war
fever, the real workmen of the Faubourgs (suburbs, workers’ districts)
came forward with public peace demonstrations so overwhelming that
Pietri, the Prefect of Police, thought it prudent to stop at once all
further street politics, on the plea that the real Paris people had given
sufficient vent to their pent-up patriotism and exuberant war
enthusiasm.

Whatever may be the incidents of Louis Bonaparte’s war with Prussia, the
death-knell of the Second Empire has already sounded at Paris. It will
end, as it began, by a parody. But let us not forget that it is the
governments and the ruling classes of Europe who enabled Louis Bonaparte
to play during 18 years the ferocious farce of the Restored Empire.

On the German side, the war is a war of defence; but who put Germany to
the necessity of defending herself? Who enabled Louis Bonaparte to wage
war upon her? Prussia! It was Bismarck who conspired with that very same
Louis Bonaparte for the purpose of crushing popular opposition at home,
and annexing Germany to the Hohenzollern dynasty. If the battle of Sadowa
had been lost instead of being won, French battalions would have overrun
Germany as the allies of Prussia. After her victory, did Prussia dream one
moment of opposing a free Germany to an enslaved France? Just the
contrary. While carefully preserving all the native beauties of her old
system, she super-added all the tricks of the Second Empire, its real
despotism, and its mock democratism, its political shams and its financial
jobs, its high-flown talk and its low legerdemains. The Bonapartist
regime, which till then only flourished on one side of the Rhine, had now
got its counterfeit on the other. From such a state of things, what else
could result but war?

If the German working class allows the present war to lose its strictly
defensive character and to degenerate into a war against the French
people, victory of defeat will prove alike disastrous. All the miseries
that befell Germany after her wars of independence will revive with
accumulated intensity.

The principles of the International are, however, too widely spread and
too firmly rooted amongst the German working class to apprehend such a sad
consummation. The voices of the French workmen had re-echoed from Germany.
A mass meeting of workmen, held at Brunswick on July 16, expressed its
full concurrence with the Paris manifesto, spurned the idea of national
antagonism to France, and wound up its resolutions with these words:

\begin{quote}

``We are the enemies of all wars, but above all of dynastic wars. ... With
deep sorrow and grief we are forced to undergo a defensive war as an
unavoidable evil; but we call, at the same time, upon the whole German
working class to render the recurrence of such an immense social
misfortune impossible by vindicating for the peoples themselves the power
to decide on peace and war, and making them masters of their own
destinies.''

\end{quote}

At Chemnitz, a meeting of delegates, representing 50,000 Saxon workmen,
adopted unanimously a resolution to this effect:

\begin{quote}

``In the name of German Democracy, and especially of the workmen forming
the Democratic Socialist Party, we declare the present war to be
exclusively dynastic.... We are happy to grasp the fraternal hand
stretched out to us by the workmen of France.... Mindful of the watchword
of the International Working Men’s Association: Proletarians of all
countries, unite, we shall never forget that the workmen of all countries
are our friends and the despots of all countries our enemies.''

\end{quote}

The Berlin branch of the International has also replied to the Paris
manifesto:

``We,'' they say, ``join with heart and hand your protestation....
Solemnly, we promise that neither the sound of the trumpets, nor the roar
of the cannon, neither victory nor defeat, shall divert us from our common
work for the union of the children of toil of all countries.''

Be it so!

In the background of this suicidal strike looms the dark figure of Russia.
It is an ominous sign that the signal for the present war should have been
given at the moment when the Moscovite government had just finished its
strategic lines of railway and was already massing troops in the direction
of the Prut. \footnote{The river Prut, rising in the southwestern Ukraine
and flowing southeast, forming part of the border between Roumania (within
an autonomous part of Austria-Hungary) and Russia (later to join the river
Danube). \emph{Length}: 853 kilometers.} Whatever sympathy the Germans may
justly claim in a war of defense against Bonapartist aggression, they
would forfeit at once by allowing the Prussian government to call for, or
accept the help of, the Cossack. Let them remember that after their war of
independence against the first Napoleon, Germany lay for generations
prostrate at the feet of the tsar.

The English working class stretch the hand of fellowship to the French and
German working people. They feel deeply convinced that whatever turn the
impending horrid war may take, the alliance of the working classes of all
countries will ultimately kill war. The very fact that while official
France and Germany are rushing into a fratricidal feud, the workmen of
France and Germany send each other messages of peace and goodwill; this
great fact, unparalleled in the history of the past, opens the vista of
a brighter future. It proves that in contrast to old society, with its
economical miseries and its political delirium, a new society is springing
up, whose International rule will be Peace, because its national ruler
will be everywhere the same – Labour! The pioneer of that new society is
the International Working Men’s Association.

\chapter{Prussian Occupation of France}

In our first manifesto of the 23rd of July, we said:

\begin{quote}

``The death-knell of the Second Empire has already sounded at Paris. It
will end, as it began, by a parody. But let us not forget that it is the
governments and the ruling classes of Europe who enabled Louis Bonaparte
to play during 18 years the ferocious farce of the Restored Empire.''

\end{quote}

Thus, even before war operations had actually set in, we treated the
Bonapartist bubble as a thing of the past.

If we were not mistaken as to the vitality of the Second Empire, we were
not wrong in our apprehension lest the German war should ``lose its
strictly defensive character and degenerate into a war against the French
people.'' The war of defense ended, in point of fact, with the surrender
of Louis Bonaparte, the Sedan capitulation, and the proclamation of the
republic at Paris. But long before these events, the very moment that the
utter rottenness of the imperialist arms became evident, the Prussian
military camarilla had resolved upon conquest. There lay an ugly obstacle
in their way – (Prussian) King William’s own proclamations at the
commencement of the war.

In a speech from the throne to the North German Diet, he had solemnly
declared to make war upon the emperor of the French and not upon the
French nation, where he said:

\begin{quote}

``The Emperor Napoleon having made by land and sea an attack on the German
nation, which desired and still desires to live in peace with the French
people, I have assumed the command of the German armies to repel his
aggression, and I have been led by military events to cross the frontiers
of France.''

\end{quote}

Not content to assert the defensive character of the war by the statement
that he only assumed the command of the German armies “to repel
aggression", he added that he was only “led by military events” to cross
the frontiers of France. A defensive war does, of course, not exclude
offensive operations, dictated by military events.

Thus, the pious king stood pledged before France and the world to
a strictly defensive war. How to release him from his solemn pledge? The
stage managers had to exhibit him as reluctantly yielding to the
irresistible behest of the German nation. They at once gave the cue to the
liberal German middle class, with its professors, its capitalists, its
aldermen, and its penmen. That middle class, which, in its struggles for
civil liberty, had, from 1846 to 1870, been exhibiting an unexampled
spectacle of irresolution, incapacity and cowardice, felt, of course,
highly delighted to bestride the European scene as the roaring lion of
German patriotism. It re-vindicated its civic independence by affecting to
force upon the Prussian government the secret designs of that same
government. It does penance for its long-continued, and almost religious,
faith in Louis Bonaparte’s infallibility, but shouting for the
dismemberment of the French republic. Let us, for a moment, listen to the
special pleadings of those stout-hearted patriots!

They dare not pretend that the people of Alsace and Lorraine pant for the
German embrace; quite the contrary. To punish their French patriotism,
Strasbourg, a town with an independent citadel commanding it, has for six
days been wantonly and fiendishly bombarded by “German” explosive shells,
setting it on fire, and killing great numbers of its defenceless
inhabitants! Yet, the soil of those provinces once upon a time belonged to
the whilom German empire. \footnote{The Holy Roman Empire of the German
nation, founded in the 10th century and constituting a union of feudal
principalities and free towns which recognized the supreme of authority of
an emperor.} Hence, it seems, the soil and the human beings grown on it
must be confiscated as imprescriptible German property. If the map of
Europe is to be re-made in the antiquary’s vein, let us by no means forget
that the Elector of Brandenburg, for his Prussian dominions, was the
vassal of the Polish republic. \footnote{In 1618 the Electorate of
Brandenburg united with the Prussian Dutchy (East Prussia), which had been
formed early in the 16th century out of the Teutonic Order possessions and
which was still a feudal vessel of the Kingdom of Poland. The Elector of
Brandenburg, a Prussian Duke at the same time, remained a Polish vassal
until 1657 when, taking advantage of Poland’s difficulties in the war
against Sweden, he secured sovereign rights to Prussian possessions.}


The more knowing patriots, however, require Alsace and the German-speaking
Lorraine as a “material guarantee” against French aggression. As this
contemptible plea has bewildered many weak-minded people, we are bound to
enter more fully upon it.

There is no doubt that the general configuration of Alsace, as compared
with the opposite bank of the Rhine, and the presence of a large fortified
town like Strasbourg, about halfway between Basle and Germersheim, very
much favour a French invasion of South Germany, while they offer peculiar
difficulties to an invasion of France from South Germany. There is,
further, no doubt that the addition of Alsace and German-speaking Lorraine
would give South Germany a much stronger frontier, inasmuch as she would
then be the master of the crest of the Vosges mountains in its whole
length, and of the fortresses which cover its northern passes. If Metz
were annexed as well, France would certainly for the moment be deprived of
her two principal bases of operation against Germany, but that would not
prevent her from concentrating a fresh one at Nancy or Verdun. While
Germany owns Coblenz, Mayence [i.e., Mainz], Germersheim, Rastatt, and
Ulm, all bases of operation against France, and plentifully made use of in
this war, with what show of fair play can she begrudge France Strasbourg
and Metz, the only two fortresses of any importance she has on that side?
Moreover, Strasbourg endangers South Germany only while South Germany is
a separate power from North Germany. From 1792 to 1795, South Germany was
never invaded from that direction, because Prussia was a party to the war
against the French Revolution; but as soon as Prussia made a peace of her
own \footnote{The Treaty of Basle concluded by Prussia, a member of the
first anti-French coalition of the European states, with the French
Republic on April 5, 1795.} in 1795, and left the South to shift for
itself, the invasions of South Germany with Strasbourg as a base began and
continued till 1809. The fact is, a united Germany can always render
Strasbourg and any French army in Alsace innocuous by concentrating all
her troops, as was done in the present war, between Saarlouis and Landau,
and advancing, or accepting battle, on the line of road between Mayence
and Metz. While the mass of the German troops is stationed there, any
French army advancing from Strasbourg into South Germany would be
outflanked, and have its communication threatened. If the present campaign
has proved anything, it is the facility of invading France from Germany.

But, in good faith, is it not altogether an absurdity and an anachronism
to make military considerations the principle by which the boundaries of
nations are to be fixed? If this rule were to prevail, Austria would still
be entitled to Venetia and the line of the Minicio, and France to the line
of the Rhine, in order to protect Paris, which lies certainly more open to
an attack from the northeast than Berlin does from the southwest. If
limits are to be fixed by military interests, there will be no end to
claims, because every military line is necessarily faulty, and may be
improved by annexing some more outlying territory; and, moreover, they can
never be fixed finally and fairly, because they always must be imposed by
the conqueror upon the conquered, and consequently carry within them the
seed of fresh wars.

Such is the lesson of all history.

Thus with nations as with individuals. To deprive them of the power of
offence, you must deprive them of the means of defence. You must not only
garrote, but murder. If every conqueror took “material guarantees" for
breaking the sinews of a nation, the first Napoleon did so by the Tilsit
Treaty, and the way he executed it against Prussia and the rest of
Germany. Yet, a few years later, his gigantic power split like a rotten
reed upon the German people. What are the “material guarantees” Prussia,
in her wildest dreams, can or dare imposes upon France, compared to the
“material guarantees” the first Napoleon had wrenched from herself? The
result will not prove the less disastrous. History will measure its
retribution, not by the intensity of the square miles conquered from
France, but by the intensity of the crime of reviving, in the second half
of the 19th century, the policy of conquest!

But, say the mouthpieces of Teutonic (German) patriotism, you must not
confound Germans with Frenchmen. What we want is not glory, but safety.
The Germans are an essentially peaceful people. In their sober
guardianship, conquest itself changes from a condition of future war into
a pledge of perpetual peace. Of course, it is not Germans that invaded
France in 1792, for the sublime purpose of bayonetting the revolution of
the 18th century. It is not Germans that befouled their hands by the
subjugation of Italy, the oppressions of Hungary, and the dismemberment of
Poland. Their present military system, which divides the whole able-bodied
male population into two parts – one standing army on service, and another
standing army on furlough, both equally bound in passive obedience to
rulers by divine right – such a military system is, of course, “a material
guarantee,” for keeping the peace and the ultimate goal of civilizing
tendencies! In Germany, as everywhere else, the sycophants of the powers
that be poison the popular mind by the incense of mendacious self-praise.

Indignant as they pretend to be at the sight of French fortresses in Metz
and Strasbourg, those German patriots see no harm in the vast system of
Moscovite fortifications at Warsaw, Modlin, and Ivangorod (All strongholds
of the Russian Empire) . While gloating at the terrors of imperialist
invasion, they blink at the infamy of autocratic tutelage.

As in 1865, promises were exchanged between Gorchakov and Bismarck. As
Louis Bonaparte flattered himself that the War of 1866, resulting in the
common exhaustion of Austria and Prussia, would make him the supreme
arbiter of Germany, so Alexander (II of Russia) flattered himself that the
War of 1870, resulting in the common exhaustion of Germany and France,
would make him the supreme arbiter of the Western continent. As the Second
Empire thought the North German Confederation incompatible with its
existence, so autocratic Russia must think herself endangered by a German
empire under Prussian leadership. Such is the law of the old political
system. Within its pale the gain of one state is the loss of the other.
The tsar’s paramount influence over Europe roots in his traditional hold
on Germany. At a moment when in Russia herself volcanic social agencies
threaten to shake the very base of autocracy, could the tsar afford to
bear with such a loss of foreign prestige? Already the Moscovite journals
repeat the language of the Bonapartist journals of the War of 1866. Do the
Teuton patriots really believe that liberty and peace will be guaranteed
to Germany by forcing France into the arms of Russia? If the fortune of
her arms, the arrogance of success, and dynastic intrigue lead Germany to
a dismemberment of French territory, there will then only remain two
courses open to her. She must at all risks become the avowed tool of
Russian aggrandizement, or, after some short respite, make again ready for
another “defensive” war, not one of those new-fangled “localized” wars,
but a war of races – a war with the Slavonic and Roman races.
\footnote{Marx’s clear assessment of Germany’s historical position took
some time to completely fulfill itself, but when it did Germany’s war on
races occurred in full force.}

The German working class have resolutely supported the war, which it was
not in their power to prevent, as a war for German independence and the
liberation of France and Europe from that pestilential incubus, the Second
Empire. It was the German workmen who, together with the rural laborers,
furnished the sinews and muscles of heroic hosts, leaving behind their
half-starved families. Decimated by the battles abroad, they will be once
more decimated by misery at home. In their turn, they are now coming
forward to ask for “guarantees” – guarantees that their immense sacrifices
have not been bought in vain, that they have conquered liberty, that the
victory over the imperialist armies will not, as in 1815, be turned into
the defeat of the German people\footnote{Marx refers here to the triumph
of feudal reaction in Germany after the downfall of Napoleon. The
feudalist unity of Germany was restored, the feudal-monarchist system was
established in the German states, which retained all the privileges of the
nobility and intensified the semi-feudal exploitation of the peasantry.};
and, as the first of these guarantees, they claim an honorable peace for
France, and the recognition of the French republic.

The Central Committee of the German Social-Democratic Workmen’s Party
issued, on September 5, a manifesto, energetically insisting upon these
guarantees.

``We,'' they say, ``protest against the annexation of Alsace and Lorraine.
And we are conscious of speaking in the name of the German working class.
In the common interest of France and Germany, in the interest of western
civilization against eastern barbarism, the German workmen will not
patiently tolerate the annexation of Alsace and Lorraine.... We shall
faithfully stand by our fellow workmen in all countries for the common
international cause of the proletariat!''

Unfortunately, we cannot feel sanguine of their immediate success. If the
French workmen amidst peace failed to stop the aggressor, are the German
workmen more likely to stop the victor amidst the clamour of arms? The
German workmen’s manifesto demands the extradition of Louis Bonaparte as
a common felon to the French republic. Their rulers are, on the contrary,
already trying hard to restore him to the Tuileries\footnote{ The
\emph{Tuileries} Palace in Paris, a residence of Napoleon III.} as the best man
to ruin France. However that may be, history will prove that the German
working class are not made of the same malleable stuff as the German
middle class. They will do their duty.

Like them, we hail the advent of the republic in France, but at the same
time we labor under misgivings which we hope will prove groundless. That
republic has not subverted the throne, but only taken its place, become
vacant. It has been proclaimed, not as a social conquest, but as
a national measure of defence. It is in the hands of a Provisional
Government composed partly of notorious Orleanists, partly of middle class
republicans, upon some of whom the insurrection of June 1848 has left its
indelible stigma. The division of labor amongst the members of that
government looks awkward. The Orleanists have seized the strongholds of
the army and the police, while to the professed republicans have fallen
the talking departments. Some of their acts go far to show that they have
inherited from the empire, not only ruins, but also its dread of the
working class. If eventual impossibilities are, in wild phraseology,
promised in the name of the republic, is it not with a view to prepare the
cry for a ``possible'' government? Is the republic, by some of its middle
class undertakers, not intended to serve as a mere stop-gap and bridge
over an Orleanist restoration?

The French working class moves, therefore, under circumstances of extreme
difficulty. Any attempt at upsetting the new government in the present
crisis, when the enemy is almost knocking at the doors of Paris, would be
a desperate folly. The French workmen must perform their duties as
citizens; but, at the same time, they must not allow themselves to be
swayed by the national souvenirs of 1792, as the French peasant allowed
themselves to be deluded by the national souvenirs of the First Empire.
They have not to recapitulate the past, but to build up the future. Let
them calmly and resolutely improve the opportunities of republican
liberty, for the work of their own class organization. It will gift them
with fresh herculean powers for the regeneration of France, and our common
task – the emancipation of labor. Upon their energies and wisdom hinges
the fate of the republic.

The English workmen have already taken measures to overcome, by
a wholesome pressure from without, the reluctance of their government to
recognize the French republic\footnote{Campaigns by English workers to
secure recognition of the French Republic proclaimed on Sept. 4, 1870. On
Sept. 5 a series of meetings and demonstrations began in London and other
big cities, at which resolutions and petitions were passed demanding that
the British Government immediately recognize the French Republic. The
General Council of the First International took a direct part in the
organization of this movement.}.The present dilatoriness of the British
government is probably intended to atone for the Anti-Jacobin war (1792)
and the former indecent haste in sanctioning the coup d’etat\footnote{
Marx is alluding to England’s active part in forming a coalition of feudal
monarchies which started a war against revolutionary France in 1792, and
also to the fact that the English oligarchy was the first in Europe to
recognize the Bonapartist regime in France, established as a result of the
coup d’etat, by Louis Bonaparte on December 2, 1851.}. The English workmen
call also upon their government to oppose by all its power the
dismemberment of France, which a part of the English press is shameless
enough to howl for. It is the same press that for 20 years deified Louis
Bonaparte as the providence of Europe, that frantically cheered on the
slaveholders’ rebellion. \footnote{During the American Civil War (1861-65)
between the industrial North and the slave-owning South, the English
bourgeois press took the side of the South.} Now, as then, it drudges for
the slaveholder.

Let the chapters of the International Working Men’s Association in every
country stir the working classes to action. If they forsake their duty, if
they remain passive, the present tremendous war will be but the harbinger
of still deadlier international feuds, and lead in every nation to
a renewed triumph over the workman by the lords of the sword, of the soil,
and of capital.

\emph{Vive la Republique!}


\chapter{France Capitulates \& the Government of Thiers}


\end{document}
